% setwd("C://aaaWork//Class Materials//MnDNR_ShortCourse//Readings//MarkRecapture//")
% Sweave("MarkRecapture.Rnw")

\documentclass[a4paper]{article}
\usepackage{amssymb,amsmath,amsfonts}       %American Math Society packages
\usepackage{color}
\usepackage[pagebackref=true]{hyperref}     %allows back reference from bibliography entry
\hypersetup{
    pdfauthor = Derek H. Ogle,
    plainpages = false,                     %front matter pages saved as page.i instead of page.1 to reduce errors
    colorlinks = true,                      %allows colored links to be used -- rather than boxes
    linkcolor = red,                        %general internal links (to equations, tables, from TOC)
    citecolor = blue,                       %color for citations
    urlcolor = magenta,                     %color for linked URLs
    bookmarksopen = false,                  %opens bookmark tree in acrobat
    breaklinks = true                       %allows long links to break onto two lines
}
\usepackage{C:/Apps/R/R-2.6.1/share/texmf/Sweave}

%==================================================================================================================================
% Overall Style Definitions --- Sizes
%==================================================================================================================================
% These are for letter paper
  \oddsidemargin = 0in    %default is 1 in, this keeps it at 1 in
  \evensidemargin = 0in
  \textheight = 9.5in

\textwidth = 6.5in
\topmargin = 0in
\headheight = 0.2in
\headsep = 0.2in

\parskip = 0.125in
\parindent = 0.0in

\newcommand{\R}[1]{      % command to put around R functions that are within a sentence
  \texttt{#1}            % Use \R{summary(lm1)}
}

\newcommand{\var}[1]{    % command to put around data variables thatare within a sentence
  \textsf{\textsl{#1}}   % Use \var{Eggs}
}

\newcommand{\dfile}[1]{  % command to put around data file names that are within a sentence
  \textsf{#1}            % Use \dfile{Eggs}
}

%defines a figure reference command so you don't have to type Figure each time
\newcommand{\figref}[1]{
  \hspace{-0.4em}\textbf{Figure \ref{#1}}\hspace{-0.3em}
}

%same as above but automatically in parentheses
\newcommand{\figrefp}[1]{
  \hspace{-0.4em}\textbf{(Figure \ref{#1})}\hspace{-0.3em}
}

%same as \figref except for tables
\newcommand{\tabref}[1]{
  \hspace{-0.4em}\textbf{Table \ref{#1}}\hspace{-0.3em}
}

%same as \figrefp except for tables
\newcommand{\tabrefp}[1]{
  \hspace{-0.4em}\textbf{(Table \ref{#1})}\hspace{-0.3em}
}

\newcommand{\sectref}[1]{
  \hspace{-0.4em}\textbf{Section \ref{#1}}\hspace{-0.3em}
}

\newcommand{\sectrefp}[1]{
  \hspace{-0.4em}\textbf{(Section \ref{#1})}\hspace{-0.3em}
}

\newenvironment{Itemize}{
\begin{itemize}
  \setlength{\topsep}{1pt}      %doesn't seem to have an effect
  \setlength{\partopsep}{1pt}   %doesn't seem to have an effect
  \setlength{\itemsep}{1pt}
  \setlength{\parskip}{0pt}     %doesn't seem to have an effect
  \setlength{\parsep}{0pt}}     %doesn't seem to have an effect
{\end{itemize}
}

\newenvironment{Enumerate}{
\begin{enumerate}
  \setlength{\topsep}{1pt}
  \setlength{\partopsep}{1pt}
  \setlength{\itemsep}{1pt}
  \setlength{\parskip}{0pt}
  \setlength{\parsep}{0pt}}
{\end{enumerate}
} 

%removes extra space in SCHUNKs
\fvset{listparameters={\setlength{\topsep}{0pt}}}
\renewenvironment{Schunk}{\vspace{\topsep}}{\vspace{\topsep}}

\usepackage{C:/Apps/R/R-2.6.1/share/texmf/Sweave}
\begin{document}




\title{Mark-Recapture Estimates of Abundance}
\date{}  % makes date blank -- title will only have the title above then
\maketitle
\vspace{-72pt}


\section{Initialization}
\begin{Schunk}
\begin{Sinput}
> library(FSA)
> setwd("C://aaaWork/Class Materials//MnDNR_ShortCourse//Readings//MarkRecapture//")
\end{Sinput}
\end{Schunk}

\section{Summarizing Capture Histories}
\subsection{Data Recorded in Capture History Format}
\begin{Schunk}
\begin{Sinput}
> data(PikeNYPartial1)
> str(PikeNYPartial1)
\end{Sinput}
\begin{Soutput}
'data.frame':	57 obs. of  5 variables:
 $ id    : int  2001 2002 2003 2004 2005 2006 2007 2008 2009 2010 ...
 $ first : int  1 1 1 1 1 1 1 1 1 1 ...
 $ second: int  0 0 0 0 0 0 0 0 0 0 ...
 $ third : int  0 0 0 0 0 0 0 0 0 0 ...
 $ fourth: int  0 0 0 0 0 0 0 0 0 0 ...
\end{Soutput}
\begin{Sinput}
> rhead(PikeNYPartial1)
\end{Sinput}
\begin{Soutput}
     id first second third fourth
46 2046     0      0     1      0
11 2011     1      0     0      0
18 2018     1      0     0      0
23 2023     1      1     0      0
55 2055     0      0     0      1
56 2056     0      0     0      1
\end{Soutput}
\begin{Sinput}
> pikech1 <- caphist.sum(PikeNYPartial1, cols = 2:5)
> str(pikech1)
\end{Sinput}
\begin{Soutput}
List of 4
 $ caphist.sum : 'table' int [, 1:10] 5 8 2 12 1 2 21 1 2 3
  ..- attr(*, "dimnames")=List of 1
  .. ..$ : chr [1:10] "0001" "0010" "0011" "0100" ...
 $ schnabel.sum:'data.frame':	4 obs. of  4 variables:
  ..$ n: int [1:4] 27 18 14 9
  ..$ m: num [1:4] 0 3 4 4
  ..$ R: num [1:4] 27 18 14 0
  ..$ M: num [1:4] 0 27 42 52
 $ methodB.top : int [1:4, 1:4] NA NA NA NA 3 NA NA NA 2 2 ...
 $ methodB.bot : num [1:4, 1:4] 0 27 27 27 3 15 18 18 4 10 ...
  ..- attr(*, "dimnames")=List of 2
  .. ..$ : chr [1:4] "m" "u" "n" "R"
  .. ..$ : NULL
 - attr(*, "class")= chr "CapHist"
\end{Soutput}
\begin{Sinput}
> pikech1$caphist.sum
\end{Sinput}
\begin{Soutput}
0001 0010 0011 0100 0101 0110 1000 1001 1010 1100
   5    8    2   12    1    2   21    1    2    3
\end{Soutput}
\begin{Sinput}
> pikech1$schnabel.sum
\end{Sinput}
\begin{Soutput}
   n m  R  M
1 27 0 27  0
2 18 3 18 27
3 14 4 14 42
4  9 4  0 52
\end{Soutput}
\end{Schunk}

\subsection{Data Recorded in Capture-by-Date Format}
\begin{Schunk}
\begin{Sinput}
> data(PikeNYPartial2)
> rhead(PikeNYPartial2)
\end{Sinput}
\begin{Soutput}
   sample   id
65 fourth 2054
48 fourth 2027
50  third 2041
7   first 2007
60  third 2051
28 second 2028
\end{Soutput}
\begin{Sinput}
> ch.tab <- table(PikeNYPartial2$id, PikeNYPartial2$sample)
> head(ch.tab)
\end{Sinput}
\begin{Soutput}
       first fourth second third
  2001     1      0      0     0
  2002     1      0      0     0
  2003     1      0      0     0
  2004     1      0      0     0
  2005     1      0      0     0
  2006     1      0      0     0
\end{Soutput}
\begin{Sinput}
> ch.df1 <- as.data.frame(ch.tab)
> rhead(ch.df1)
\end{Sinput}
\begin{Soutput}
    Var1   Var2 Freq
167 2053 second    0
222 2051  third    1
213 2042  third    0
204 2033  third    0
202 2031  third    0
78  2021 fourth    0
\end{Soutput}
\begin{Sinput}
> ch.df2 <- unstack(ch.df1, Freq ~ Var2)
> rhead(ch.df2)
\end{Sinput}
\begin{Soutput}
   first fourth second third
57     0      1      0     0
34     0      0      1     0
35     0      0      1     0
3      1      0      0     0
12     1      0      0     0
39     0      0      1     0
\end{Soutput}
\begin{Sinput}
> ch.df2 <- ch.df2[, c(1, 3, 4, 2)]
> rhead(ch.df2)
\end{Sinput}
\begin{Soutput}
   first second third fourth
33     0      1     0      0
26     1      0     1      0
38     0      1     0      0
7      1      0     0      0
4      1      0     0      0
28     0      1     0      0
\end{Soutput}
\begin{Sinput}
> pikech2 <- caphist.sum(ch.df2)
> pikech2$caphist.sum
\end{Sinput}
\begin{Soutput}
0001 0010 0011 0100 0101 0110 1000 1001 1010 1100
   5    8    2   12    1    2   21    1    2    3
\end{Soutput}
\end{Schunk}


\section{Closed Population Single Sample -- Jackson Lake Bluegill}
\subsection{Petersen Method}
\begin{Schunk}
\begin{Sinput}
> data(BluegillJL)
> str(BluegillJL)
\end{Sinput}
\begin{Soutput}
'data.frame':	277 obs. of  2 variables:
 $ first : int  1 0 1 0 1 1 1 1 1 1 ...
 $ second: int  0 1 0 1 0 0 0 0 0 0 ...
\end{Soutput}
\begin{Sinput}
> rhead(BluegillJL)
\end{Sinput}
\begin{Soutput}
    first second
80      0      1
269     1      0
81      1      0
117     1      0
270     1      0
110     1      0
\end{Soutput}
\begin{Sinput}
> bgch <- caphist.sum(BluegillJL)
> bgch$caphist.sum
\end{Sinput}
\begin{Soutput}
 01  10  11
 81 187   9
\end{Soutput}
\begin{Sinput}
> mr1 <- mr.closed1(196, 90, 9)
> summary(mr1)
\end{Sinput}
\begin{Soutput}
Used the 'naive' Petersen method.
Observed inputs of: M=196, n=90, and m=9.
Resulted in a population estimate (N) of 1960.
\end{Soutput}
\begin{Sinput}
> confint(mr1)
\end{Sinput}
\begin{Soutput}
The Poisson method was used to construct the CI for N.
 95% LCI 95% UCI
   975.4  3448.4
\end{Soutput}
\begin{Sinput}
> confint(mr1, citype = "hypergeom")
\end{Sinput}
\begin{Soutput}
The hypergeom method was used to construct the CI for N.
 95% LCI 95% UCI
    1101    4151
\end{Soutput}
\end{Schunk}

\subsection{Chapman Modification}
\begin{Schunk}
\begin{Sinput}
> mr2 <- mr.closed1(196, 90, 9, type = "C")
> summary(mr2)
\end{Sinput}
\begin{Soutput}
Used Chapman's modification of the Petersen method.
Observed inputs of: M=196, n=90, and m=9.
Resulted in a population estimate (N) of 1792.
\end{Soutput}
\begin{Sinput}
> confint(mr2)
\end{Sinput}
\begin{Soutput}
The Poisson method was used to construct the CI for N.
 95% LCI 95% UCI
   990.3  3503.5
\end{Soutput}
\end{Schunk}

\section{Closed Population Multiple Samples -- New York Pike}
\subsection{Capture Summaries Already Known}
\begin{Schunk}
\begin{Sinput}
> n <- c(27, 18, 14, 9)
> m <- c(0, 3, 4, 4)
> R <- c(27, 18, 14, 0)
> ex1 <- mr.closed2(n, m, R, type = "S")
> summary(ex1)
\end{Sinput}
\begin{Soutput}
Used the Schnabel method with Chapman modification.
Resulted in a population estimate (N) of 128.
\end{Soutput}
\begin{Sinput}
> confint(ex1)
\end{Sinput}
\begin{Soutput}
The Poisson method was used to construct the CI for N.
 95% LCI 95% UCI
    74.6   237.6
\end{Soutput}
\end{Schunk}

\subsection{Capture Histories were Recorded}
This (re)uses the the \R{PikeNYPartial1} data frame and \R{pikech1} object introduced in a previous section.
\begin{Schunk}
\begin{Sinput}
> ex2 <- mr.closed2(pikech1)
> summary(ex2)
\end{Sinput}
\begin{Soutput}
Used the Schnabel method with Chapman modification.
Resulted in a population estimate (N) of 128.
\end{Soutput}
\begin{Sinput}
> confint(ex2)
\end{Sinput}
\begin{Soutput}
The Poisson method was used to construct the CI for N.
 95% LCI 95% UCI
    74.6   237.6
\end{Soutput}
\begin{Sinput}
> plot(ex2, loess = TRUE)
\end{Sinput}
\end{Schunk}
\includegraphics[width=3in]{Figs/mr-Schnabel1}

\subsection{Schumacher-Eschmeyer Method}
\begin{Schunk}
\begin{Sinput}
> ex3 <- mr.closed2(pikech1, type = "SE")
> summary(ex3)
\end{Sinput}
\begin{Soutput}
Used the Schumacher-Eschmeyer method.
Resulted in a population estimate (N) of 136.
\end{Soutput}
\begin{Sinput}
> confint(ex3)
\end{Sinput}
\begin{Soutput}
The normal method was used to construct the CI for N.
 95% LCI 95% UCI
    96.6   229.8
\end{Soutput}
\end{Schunk}

\section{Open Population Multiple Samples}
\subsection{Capture Histories Given}
\begin{Schunk}
\begin{Sinput}
> data(CutthroatAL)
> rhead(CutthroatAL)
\end{Sinput}
\begin{Soutput}
     ID first second third
277 277     0      1     0
56   56     0      0     1
328 328     0      1     0
310 310     0      1     0
11   11     0      0     1
81   81     0      0     1
\end{Soutput}
\begin{Sinput}
> ch1 <- caphist.sum(CutthroatAL, cols = -1)
> ch1$methodB.top
\end{Sinput}
\begin{Soutput}
     [,1] [,2] [,3]
[1,]   NA   17    4
[2,]   NA   NA   75
[3,]   NA   NA   NA
\end{Soutput}
\begin{Sinput}
> ch1$methodB.bot
\end{Sinput}
\begin{Soutput}
  [,1] [,2] [,3]
m    0   17   79
u   75  244  198
n   75  261  277
R   75  261    0
\end{Soutput}
\begin{Sinput}
> ex1 <- mr.open(ch1)
> summary(ex1)
\end{Sinput}
\begin{Soutput}
Observables
   m   n   R  r  z
1  0  75  75 21 NA
2 17 261 261 75  4
3 79 277   0 NA NA

Estimates
     M M.se     N  N.se   phi phi.se  B B.se
1   NA   NA    NA    NA 0.411  0.098 NA   NA
2 30.8    6 448.2 108.2    NA     NA NA   NA
3   NA   NA    NA    NA    NA     NA NA   NA

Standard error of phi includes sampling and individual variability.
\end{Soutput}
\begin{Sinput}
> confint(ex1)
\end{Sinput}
\begin{Soutput}
The Jolly method was used to construct confidence intervals.

  N.lci N.uci phi.lci phi.uci B.lci B.uci
1    NA    NA   0.218   0.603    NA    NA
2 236.1 660.2      NA      NA    NA    NA
3    NA    NA      NA      NA    NA    NA
\end{Soutput}
\end{Schunk}

\subsection{Summaries Known, Must Be Entered into Vectors}
\begin{Schunk}
\begin{Sinput}
> s1 <- rep(NA, 5)
> s2 <- c(6, rep(NA, 4))
> s3 <- c(7, 3, rep(NA, 3))
> s4 <- c(7, 4, 6, NA, NA)
> s5 <- c(4, 3, 6, 9, NA)
> mb.top <- cbind(s1, s2, s3, s4, s5)
> mb.top
\end{Sinput}
\begin{Soutput}
     s1 s2 s3 s4 s5
[1,] NA  6  7  7  4
[2,] NA NA  3  4  3
[3,] NA NA NA  6  6
[4,] NA NA NA NA  9
[5,] NA NA NA NA NA
\end{Soutput}
\begin{Sinput}
> m <- c(0, 6, 10, 17, 22)
> u <- c(28, 21, 22, 17, 11)
> n <- c(28, 27, 32, 34, 33)
> R <- c(28, 27, 32, 34, 0)
> mb.bot <- rbind(m, u, n, R)
> mb.bot
\end{Sinput}
\begin{Soutput}
  [,1] [,2] [,3] [,4] [,5]
m    0    6   10   17   22
u   28   21   22   17   11
n   28   27   32   34   33
R   28   27   32   34    0
\end{Soutput}
\begin{Sinput}
> ex2 <- mr.open(mb.top, mb.bot)
> summary(ex2)
\end{Sinput}
\begin{Soutput}
Observables
   m  n  R  r  z
1  0 28 28 24 NA
2  6 27 27 10 18
3 10 32 32 12 18
4 17 34 34  9 13
5 22 33  0 NA NA

Estimates
     M M.se     N N.se   phi phi.se     B B.se
1   NA   NA    NA   NA 1.851  0.460    NA   NA
2 51.8 14.5 207.3 89.3 0.765  0.246   8.6 66.8
3 55.7 13.6 167.1 56.1 0.804  0.266 -12.9 37.4
4 62.5 17.2 121.5 37.5    NA     NA    NA   NA
5   NA   NA    NA   NA    NA     NA    NA   NA

Standard error of phi includes sampling and individual variability.
\end{Soutput}
\begin{Sinput}
> confint(ex2)
\end{Sinput}
\begin{Soutput}
The Jolly method was used to construct confidence intervals.

  N.lci N.uci phi.lci phi.uci  B.lci B.uci
1    NA    NA   0.949   2.752     NA    NA
2  32.2 382.3   0.283   1.247 -122.4 139.5
3  57.2 277.0   0.283   1.326  -86.3  60.5
4  47.9 195.1      NA      NA     NA    NA
5    NA    NA      NA      NA     NA    NA
\end{Soutput}
\end{Schunk}

\section{Analyses with \R{Rcapture} Package}
The ``program MARK'' is the industry standard for analyze capture-recapture data in a log-linear framework.  This software depends on data organized in capture history format as illustrated in these notes.  Recently the \R{Rcapture} package was introduced as a means of fitting the log-linear capture-recapture models in R.  This package is available for download from CRAN and an article describing how to use \R{Rcapture} can be found in the April 2007 issue of the on-line Journal of Statistical Software (\url{http://www.jstatsoft.org}).

For those of you familiar with ``program MARK'' you may recognize some of the output below.

\begin{Schunk}
\begin{Sinput}
> library(Rcapture)
> d <- descriptive(PikeNYPartial1[, 2:5])
> plot(d)
\end{Sinput}
\end{Schunk}
\includegraphics[width=5in]{Figs/mr-Rcapture1}

\begin{Schunk}
\begin{Sinput}
> r <- closedp(PikeNYPartial1[, 2:5])
> r
\end{Sinput}
\begin{Soutput}
Number of captured units: 57

Abundance estimations and model fits:
              abundance  stderr  deviance  df     AIC
M0                145.7    35.7    15.756  13  51.235
Mt                139.0    33.5     4.201  10  45.680
Mh Chao           145.7    35.7    15.756  13  51.235
Mh Poisson2        57.0    20.2    13.752  12  51.230
Mh Darroch         57.0     0.0    13.752  12  51.230
Mth Chao          139.0    33.5     4.201  10  45.680
Mth Poisson2       57.0    19.5     2.310   9  45.788
Mth Darroch        57.0     0.0     2.310   9  45.788
Mb                 64.5     5.4     4.271  12  41.750
Mbh                64.9     8.5     4.265  11  43.744

Note: 2 eta parameters has been set to zero in the Mh Chao model
Note: 2 eta parameters has been set to zero in the Mth Chao model
\end{Soutput}
\begin{Sinput}
> r1 <- openp(PikeNYPartial1[, 2:5])
> r1
\end{Sinput}
\begin{Soutput}
Model fit:
              deviance      df        AIC
fitted model     2.683       7     50.162

Test for trap effect:
                                   deviance      df        AIC
model with homogenous trap effect     2.683       6     52.162

Capture probabilities:
          estimate  stderr
period 1        --      --
period 2    0.1429  0.0830
period 3    0.2222  0.1821
period 4        --      --

Survival probabilities:
               estimate  stderr
period 1 -> 2    0.7778  0.4647
period 2 -> 3    0.5000  0.4098
period 3 -> 4        --      --

Abundances:
          estimate  stderr
period 1        --      --
period 2       126    67.9
period 3        63    49.5
period 4        --      --

Number of new arrivals:
               estimate  stderr
period 1 -> 2        --      --
period 2 -> 3         0       0
period 3 -> 4        --      --

Total number of units who ever inhabited the survey area:
             estimate  stderr
all periods       132    56.1

Total number of captured units: 57

Note: 1 gamma parameter has been set to zero
\end{Soutput}
\end{Schunk}

\end{document}
